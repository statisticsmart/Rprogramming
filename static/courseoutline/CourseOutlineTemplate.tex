%%%%%%%%%%%%%%%%%%%%%%%%%%%%%%%%%%%%%%%%%%%%%%%%%%%%%
%%%%%	 		Course Outline- Template		%%%%%
%%%%%		  Prepaired by Dr. R.M. Silva		%%%%%
%%%%%			Department of Statistics		%%%%%
%%%%%		  Faculty of Applied Sciences		%%%%%
%%%%%		University of Sri Jayewardenepura   %%%%%
%%%%%				Dec. 09th, 2017				%%%%%
%%%%%%%%%%%%%%%%%%%%%%%%%%%%%%%%%%%%%%%%%%%%%%%%%%%%%


% Please place the SJPLogo in the same folder that you are going to run this Latex file

\documentclass[a4paper,12pt]{article}
\usepackage{graphicx}
\usepackage{enumitem}
\usepackage{setspace}
\usepackage{multicol} 
\usepackage{color}
\usepackage[
top    = 2.00cm,
bottom = 2.00cm,
left   = 2.00cm,
right  = 2.00cm]{geometry}
\usepackage{hyperref}
\hypersetup{pdfstartview={XYZ null null 1.00}}

\begin{document}

\begin{figure}[ht]
	\begin{center}
		\includegraphics[angle=0,scale=0.05]{SJPLogo.jpg}
	\end{center}
\end{figure}

\vspace{-1cm}

\begin{center}
	University of Sri Jayewardenepura\\
	Faculty of Applied Scineces \\
	Department of Statistics
\end{center}


\noindent\rule{17cm}{0.4pt} 	%This will create a horizontal line

\begin{multicols}{3}
	\noindent\textbf{Batch: 2016/2017}
	
	\columnbreak
	\noindent\textbf{Year: 2018 }
	
	\columnbreak
	\noindent\textbf{Semester: First Semester}
	
\end{multicols}

\noindent\rule{17cm}{0.4pt}	% horizontal line

\vspace{0.5cm}
\noindent\textbf{Course Unit:}\\

\noindent\textbf{Type of the course unit:}\\

\noindent\textbf{Pre-Requisites:}\\
\noindent\textit{{\scriptsize ( Eg: STA 114 2.0 Probability and Distribution Theory I. If there are no pre-requisites, please mention ``None''. )}}\\

\noindent\textbf{Workload:}\\
\noindent\textit{{\scriptsize ( Eg: 100 learning hours. This includes approximately 30 hours of lectures and additional time spent by the student on self-learning, assignments and assessments.)}}\\

\noindent\textbf{Course Objective(s):}
\begin{itemize}
	\setlength\itemsep{0.1mm}
	\item First objective
	\item Second objective
\end{itemize}
\noindent\textbf{Course Contents:}
\begin{itemize}
	\setlength\itemsep{0.1mm}
	\item First content: first sub content, second sub content; first component, second component 
	\item Second content 
\end{itemize}

\noindent\textbf{Learning Outcomes:}
At the end of this course, the student should be able to:
\begin{itemize}
	\setlength\itemsep{0.1mm}
	\item First learning outcome
	\item Second learning outcome
\end{itemize}

\noindent\textbf{Method of Assessment:}
\begin{itemize}
	\setlength\itemsep{0.1mm}
	\item Continuous assessments: 30\%
	\item Final examination: 70\%
\end{itemize}

\noindent\textbf{Recommended Readings:}
\begin{itemize}
	\setlength\itemsep{0.1mm}
	\item Title: \\
		  Author(s): \\
		  Publisher:
	\item Second Book
\end{itemize}

\noindent\textbf{Lecturer in Charge:}

\begin{itemize}
	\setlength\itemsep{0.1mm}
	\item[] Name: 
	\item[] Email:
\end{itemize}


\textit{\color{red}Note: If there are more than two pages in the course outline, use two columns per page.}

\end{document}
