\PassOptionsToPackage{unicode=true}{hyperref} % options for packages loaded elsewhere
\PassOptionsToPackage{hyphens}{url}
%
\documentclass[]{article}
\usepackage{lmodern}
\usepackage{amssymb,amsmath}
\usepackage{ifxetex,ifluatex}
\usepackage{fixltx2e} % provides \textsubscript
\ifnum 0\ifxetex 1\fi\ifluatex 1\fi=0 % if pdftex
  \usepackage[T1]{fontenc}
  \usepackage[utf8]{inputenc}
  \usepackage{textcomp} % provides euro and other symbols
\else % if luatex or xelatex
  \usepackage{unicode-math}
  \defaultfontfeatures{Ligatures=TeX,Scale=MatchLowercase}
\fi
% use upquote if available, for straight quotes in verbatim environments
\IfFileExists{upquote.sty}{\usepackage{upquote}}{}
% use microtype if available
\IfFileExists{microtype.sty}{%
\usepackage[]{microtype}
\UseMicrotypeSet[protrusion]{basicmath} % disable protrusion for tt fonts
}{}
\IfFileExists{parskip.sty}{%
\usepackage{parskip}
}{% else
\setlength{\parindent}{0pt}
\setlength{\parskip}{6pt plus 2pt minus 1pt}
}
\usepackage{hyperref}
\hypersetup{
            pdfborder={0 0 0},
            breaklinks=true}
\urlstyle{same}  % don't use monospace font for urls
\usepackage[margin=1in]{geometry}
\usepackage{color}
\usepackage{fancyvrb}
\newcommand{\VerbBar}{|}
\newcommand{\VERB}{\Verb[commandchars=\\\{\}]}
\DefineVerbatimEnvironment{Highlighting}{Verbatim}{commandchars=\\\{\}}
% Add ',fontsize=\small' for more characters per line
\usepackage{framed}
\definecolor{shadecolor}{RGB}{248,248,248}
\newenvironment{Shaded}{\begin{snugshade}}{\end{snugshade}}
\newcommand{\AlertTok}[1]{\textcolor[rgb]{0.94,0.16,0.16}{#1}}
\newcommand{\AnnotationTok}[1]{\textcolor[rgb]{0.56,0.35,0.01}{\textbf{\textit{#1}}}}
\newcommand{\AttributeTok}[1]{\textcolor[rgb]{0.77,0.63,0.00}{#1}}
\newcommand{\BaseNTok}[1]{\textcolor[rgb]{0.00,0.00,0.81}{#1}}
\newcommand{\BuiltInTok}[1]{#1}
\newcommand{\CharTok}[1]{\textcolor[rgb]{0.31,0.60,0.02}{#1}}
\newcommand{\CommentTok}[1]{\textcolor[rgb]{0.56,0.35,0.01}{\textit{#1}}}
\newcommand{\CommentVarTok}[1]{\textcolor[rgb]{0.56,0.35,0.01}{\textbf{\textit{#1}}}}
\newcommand{\ConstantTok}[1]{\textcolor[rgb]{0.00,0.00,0.00}{#1}}
\newcommand{\ControlFlowTok}[1]{\textcolor[rgb]{0.13,0.29,0.53}{\textbf{#1}}}
\newcommand{\DataTypeTok}[1]{\textcolor[rgb]{0.13,0.29,0.53}{#1}}
\newcommand{\DecValTok}[1]{\textcolor[rgb]{0.00,0.00,0.81}{#1}}
\newcommand{\DocumentationTok}[1]{\textcolor[rgb]{0.56,0.35,0.01}{\textbf{\textit{#1}}}}
\newcommand{\ErrorTok}[1]{\textcolor[rgb]{0.64,0.00,0.00}{\textbf{#1}}}
\newcommand{\ExtensionTok}[1]{#1}
\newcommand{\FloatTok}[1]{\textcolor[rgb]{0.00,0.00,0.81}{#1}}
\newcommand{\FunctionTok}[1]{\textcolor[rgb]{0.00,0.00,0.00}{#1}}
\newcommand{\ImportTok}[1]{#1}
\newcommand{\InformationTok}[1]{\textcolor[rgb]{0.56,0.35,0.01}{\textbf{\textit{#1}}}}
\newcommand{\KeywordTok}[1]{\textcolor[rgb]{0.13,0.29,0.53}{\textbf{#1}}}
\newcommand{\NormalTok}[1]{#1}
\newcommand{\OperatorTok}[1]{\textcolor[rgb]{0.81,0.36,0.00}{\textbf{#1}}}
\newcommand{\OtherTok}[1]{\textcolor[rgb]{0.56,0.35,0.01}{#1}}
\newcommand{\PreprocessorTok}[1]{\textcolor[rgb]{0.56,0.35,0.01}{\textit{#1}}}
\newcommand{\RegionMarkerTok}[1]{#1}
\newcommand{\SpecialCharTok}[1]{\textcolor[rgb]{0.00,0.00,0.00}{#1}}
\newcommand{\SpecialStringTok}[1]{\textcolor[rgb]{0.31,0.60,0.02}{#1}}
\newcommand{\StringTok}[1]{\textcolor[rgb]{0.31,0.60,0.02}{#1}}
\newcommand{\VariableTok}[1]{\textcolor[rgb]{0.00,0.00,0.00}{#1}}
\newcommand{\VerbatimStringTok}[1]{\textcolor[rgb]{0.31,0.60,0.02}{#1}}
\newcommand{\WarningTok}[1]{\textcolor[rgb]{0.56,0.35,0.01}{\textbf{\textit{#1}}}}
\usepackage{longtable,booktabs}
% Fix footnotes in tables (requires footnote package)
\IfFileExists{footnote.sty}{\usepackage{footnote}\makesavenoteenv{longtable}}{}
\usepackage{graphicx,grffile}
\makeatletter
\def\maxwidth{\ifdim\Gin@nat@width>\linewidth\linewidth\else\Gin@nat@width\fi}
\def\maxheight{\ifdim\Gin@nat@height>\textheight\textheight\else\Gin@nat@height\fi}
\makeatother
% Scale images if necessary, so that they will not overflow the page
% margins by default, and it is still possible to overwrite the defaults
% using explicit options in \includegraphics[width, height, ...]{}
\setkeys{Gin}{width=\maxwidth,height=\maxheight,keepaspectratio}
\setlength{\emergencystretch}{3em}  % prevent overfull lines
\providecommand{\tightlist}{%
  \setlength{\itemsep}{0pt}\setlength{\parskip}{0pt}}
\setcounter{secnumdepth}{0}
% Redefines (sub)paragraphs to behave more like sections
\ifx\paragraph\undefined\else
\let\oldparagraph\paragraph
\renewcommand{\paragraph}[1]{\oldparagraph{#1}\mbox{}}
\fi
\ifx\subparagraph\undefined\else
\let\oldsubparagraph\subparagraph
\renewcommand{\subparagraph}[1]{\oldsubparagraph{#1}\mbox{}}
\fi

% set default figure placement to htbp
\makeatletter
\def\fps@figure{htbp}
\makeatother


\author{}
\date{\vspace{-2.5em}}

\begin{document}

\hypertarget{hypotheses-testing}{%
\subsection{Hypotheses testing}\label{hypotheses-testing}}

\hypertarget{question-1}{%
\subsubsection{Question 1}\label{question-1}}

A national research institute in Sri Lanka concludes that Sri Lankans
watch television on average 25 hours per week. It seems likely that
graduate students do not watch nearly this much television per week. To
test this, the following data were gathered from a random sample of 50
graduate students.

\begin{Shaded}
\begin{Highlighting}[]
\NormalTok{tvhour <-}\StringTok{ }\KeywordTok{c}\NormalTok{(}\DecValTok{24}\NormalTok{, }\DecValTok{20}\NormalTok{, }\FloatTok{29.3}\NormalTok{, }\FloatTok{25.1}\NormalTok{, }\FloatTok{30.6}\NormalTok{, }\FloatTok{34.6}\NormalTok{, }\FloatTok{30.0}\NormalTok{, }\FloatTok{39.0}\NormalTok{, }\FloatTok{33.7}\NormalTok{, }\FloatTok{31.6}\NormalTok{, }
       \FloatTok{25.9}\NormalTok{, }\FloatTok{34.4}\NormalTok{, }\FloatTok{26.9}\NormalTok{, }\FloatTok{23.0}\NormalTok{, }\FloatTok{31.1}\NormalTok{, }\FloatTok{29.3}\NormalTok{, }\FloatTok{34.5}\NormalTok{, }\FloatTok{35.1}\NormalTok{, }\FloatTok{31.2}\NormalTok{, }\FloatTok{33.2}\NormalTok{, }
       \FloatTok{30.2}\NormalTok{, }\FloatTok{36.4}\NormalTok{, }\FloatTok{37.5}\NormalTok{, }\FloatTok{27.6}\NormalTok{, }\FloatTok{24.6}\NormalTok{, }\FloatTok{23.9}\NormalTok{, }\FloatTok{27.0}\NormalTok{, }\FloatTok{29.5}\NormalTok{, }\FloatTok{30.1}\NormalTok{, }\FloatTok{29.6}\NormalTok{, }
       \FloatTok{27.3}\NormalTok{, }\FloatTok{31.2}\NormalTok{, }\FloatTok{32.5}\NormalTok{, }\FloatTok{25.7}\NormalTok{, }\FloatTok{30.1}\NormalTok{, }\FloatTok{24.2}\NormalTok{, }\FloatTok{24.1}\NormalTok{, }\FloatTok{26.4}\NormalTok{, }\FloatTok{31.0}\NormalTok{, }\FloatTok{20.7}\NormalTok{, }
       \FloatTok{33.5}\NormalTok{, }\FloatTok{32.2}\NormalTok{, }\FloatTok{34.7}\NormalTok{, }\FloatTok{32.6}\NormalTok{, }\FloatTok{33.5}\NormalTok{, }\FloatTok{32.7}\NormalTok{, }\FloatTok{25.6}\NormalTok{, }\FloatTok{31.1}\NormalTok{, }\FloatTok{32.9}\NormalTok{, }\FloatTok{25.9}\NormalTok{)}
\end{Highlighting}
\end{Shaded}

\begin{enumerate}
\def\labelenumi{\roman{enumi}.}
\item
  State the appropriate null and alternative hypothesis.
\item
  Perform an analysis to test the hypothesis in (i).
\end{enumerate}

\hypertarget{question-2}{%
\subsubsection{Question 2}\label{question-2}}

The following data (in litres) which were selected randomly from a
normally distributed population of values, represent measurements of a
bottle content that is supposed to contain, on average 5L.

\begin{Shaded}
\begin{Highlighting}[]
\NormalTok{bottle.content <-}\StringTok{ }\KeywordTok{c}\NormalTok{(}\FloatTok{5.1}\NormalTok{, }\FloatTok{5.4}\NormalTok{, }\FloatTok{5.3}\NormalTok{, }\FloatTok{5.2}\NormalTok{, }\FloatTok{5.5}\NormalTok{, }\FloatTok{5.6}\NormalTok{, }\FloatTok{5.4}\NormalTok{, }\FloatTok{5.3}\NormalTok{, }\FloatTok{5.4}\NormalTok{, }\FloatTok{5.2}\NormalTok{,}
                    \FloatTok{5.8}\NormalTok{, }\FloatTok{5.2}\NormalTok{, }\FloatTok{5.2}\NormalTok{, }\FloatTok{5.3}\NormalTok{, }\FloatTok{5.1}\NormalTok{, }\FloatTok{5.3}\NormalTok{, }\FloatTok{5.4}\NormalTok{, }\FloatTok{5.5}\NormalTok{, }\FloatTok{5.5}\NormalTok{, }\FloatTok{5.7}\NormalTok{)}
\end{Highlighting}
\end{Shaded}

Use these data and \(\alpha=0.01\) to test the hypothesis that the
content average 5L.

\hypertarget{question-3}{%
\paragraph{Question 3}\label{question-3}}

Use the following data to construct 80\%, 90\% and 95\% confidence
intervals to estimate \(\mu\).

\begin{verbatim}
 [1] 103.4 102.7  90.1  89.8  74.8 106.5 108.5  97.9 115.8 101.1  92.3 108.2
[13] 110.8  96.7  98.9 116.2 115.3  72.7  79.4 100.5
\end{verbatim}

Plot all confidence intervals on a single graph plane.

\hypertarget{question-4}{%
\paragraph{Question 4}\label{question-4}}

A mathematics teacher gives their students a calculus pretest on the
first day of class, and a similar test towards the end of the course.
The results are shown below.

\begin{longtable}[]{@{}ll@{}}
\toprule
before & after\tabularnewline
\midrule
\endhead
72 & 65\tabularnewline
57 & 91\tabularnewline
71 & 67\tabularnewline
64 & 66\tabularnewline
55 & 85\tabularnewline
60 & 60\tabularnewline
67 & 97\tabularnewline
65 & 72\tabularnewline
84 & 54\tabularnewline
64 & 65\tabularnewline
56 & 70\tabularnewline
60 & 75\tabularnewline
65 & 75\tabularnewline
69 & 70\tabularnewline
75 & 72\tabularnewline
67 & 78\tabularnewline
74 & 75\tabularnewline
81 & 90\tabularnewline
80 & 85\tabularnewline
71 & 76\tabularnewline
68 & 65\tabularnewline
70 & 83\tabularnewline
\bottomrule
\end{longtable}

Determine whether the students performed significantly better on the
post test, using \(\alpha=0.05\)

\hypertarget{question-5}{%
\paragraph{Question 5}\label{question-5}}

Try to solve your Statistical Inference course hypotheses testing
problems using R.

\end{document}
