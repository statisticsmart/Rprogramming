%%%%%%%%%%%%%%%%%%%%%%%%%%%%%%%%%%%%%%%%%%%%%%%%%%%%%
%%%%%	 		Course Outline- Template		%%%%%
%%%%%		  Prepaired by Dr. R.M. Silva		%%%%%
%%%%%			Department of Statistics		%%%%%
%%%%%		  Faculty of Applied Sciences		%%%%%
%%%%%		University of Sri Jayewardenepura   %%%%%
%%%%%				Dec. 09th, 2017				%%%%%
%%%%%%%%%%%%%%%%%%%%%%%%%%%%%%%%%%%%%%%%%%%%%%%%%%%%%


% Please place the SJPLogo in the same folder that you are going to run this Latex file

\documentclass[a4paper,12pt]{article}
\usepackage{graphicx}
\usepackage{enumitem}
\usepackage{setspace}
\usepackage{multicol} 
\usepackage{color}
\usepackage[
top    = 2.00cm,
bottom = 2.00cm,
left   = 2.00cm,
right  = 2.00cm]{geometry}
\usepackage{hyperref}
\hypersetup{pdfstartview={XYZ null null 1.00}}

\begin{document}

\begin{figure}[ht]
	\begin{center}
		\includegraphics[angle=0,scale=0.05]{SJPLogo.jpg}
	\end{center}
\end{figure}

\vspace{-1cm}

\begin{center}
	University of Sri Jayewardenepura\\
	Faculty of Applied Scineces \\
	Department of Statistics
\end{center}


\noindent\rule{17cm}{0.4pt} 	%This will create a horizontal line

\begin{multicols}{3}
	\noindent\textbf{Batch: 2016/2017}
	
	\columnbreak
	\noindent\textbf{Year: 2020 }
	
	\columnbreak
	\noindent\textbf{Semester: First Semester}
	
\end{multicols}

\noindent\rule{17cm}{0.4pt}	% horizontal line

\vspace{0.5cm}
\noindent\textbf{Course Unit: STA 326 2.0 Programming and Data Analysis with R}\\

\noindent\textbf{Type of the course unit: Core for special degree students/ Optional for others}\\

\noindent\textbf{Pre-Requisites:}\\
\noindent\textit{{ STA 114 2.0 Probability and Distribution Theory I, STA 123 2.0 Probability and Distribution Theory II, STA 124 1.5 Data Analysis I, STA 213 2.0 Statistical Inference, STA 226 1.5 Data Analysis II}}\\

\noindent\textbf{Workload:}\\
\noindent\textit{{ Minimum total expected workload to achieve the learning outcomes for this unit is 100 hours per semester typically comprising a mixture of scheduled learning activities, independent study and 30 hours of lectures. Independent study may include associated readings, assessment and preparation for scheduled activities.}}\\

\noindent\textbf{Course Objective(s):}
\begin{itemize}
	\setlength\itemsep{0.1mm}
	\item To introduce R programming for data science applications.
	\item To introduce how data is communicated to make impactful decisions.
\end{itemize}
\noindent\textbf{Course Contents:}
\begin{enumerate}[label*=\arabic*.]
	\setlength\itemsep{0.1mm}
	\item R basics: Objects in R, Data types, Operations, Installing packages, Control structures, Piping
	\item Writing functions in R
	\item Data analysis with the tidyverse
	\begin{enumerate}[label*=\arabic*.]
	\item Data import and export
	\item Data wrangling: Tidy data principles, Reshaping data into tidy form, Data transformation
	\item Data visualization: The grammar of graphics
	\item Statistical	modelling	and	inference
	\item Communication: Dynamic reproducible reporting
	\end{enumerate}
\end{enumerate}

\newpage

\noindent\textbf{Learning Outcomes:}
At the end of this course, the student should be able to:
\begin{itemize}
	\setlength\itemsep{0.1mm}
\item download and install R and R Studio
\item navigate and optimise the R integrated development environment (IDE) R Studio.
\item install and load add-in packages.
\item use Tidyverse packages in data science workflow 
\item import external data into R for data processing and statistical analysis.
\item fluently reshape complex, messy, data into the most convenient form for analysis or reporting.
\item produce effective visualisation and modelling to understand relationships between variables, and make decisions with data
\item interpret the results of analysis and communicate these to a broad audience.

\end{itemize}

\noindent\textbf{Method of Assessment:}
\begin{itemize}
	\setlength\itemsep{0.1mm}
	\item Mid-semester examination: 20\%
	\item Final examination: 80\%
\end{itemize}

\noindent\textbf{Recommended Readings:}
\begin{itemize}
	\setlength\itemsep{0.1mm}
	\item Course website: Everything you want to know about the course, and everything you will need for the course (links to weekly reading, tutorials and lecture materials) will be posted at \textcolor{blue}{hellor.netlify.com}\\
	Author: Thiyanga Talagala
	\item Title: R for Data Science \\
		  Author(s): Hadley Wickham and Garrett Grolemund \\
		  Publisher: O'REILLY \\
		  This book is available online for free. Visit \textcolor{blue}{https://r4ds.had.co.nz/}
		\item Title: Advanced R \\
		  Author(s): Hadley Wickham \\
		  Publisher: Chapman \& Hall/CRC \\
		  This book is available online for free. Visit \textcolor{blue}{https://adv-r.hadley.nz/}
\end{itemize}

\noindent\textbf{Lecturer in Charge:}

\begin{itemize}
	\setlength\itemsep{0.1mm}
	\item[] Name: Dr Thiyanga  Talagala
	\item[] Email: ttalagala@sjp.ac.lk
\end{itemize}


\noindent\textbf{\textit{Note: The course can be dropped on or before 13 February 2020.}}

\end{document}
