%%%%%%%%%%%%%%%%%%%%%%%%%%%%%%%%%%%%%%%%%%%%%%%%%%%%%
%%%%%	 		Course Outline- Template		%%%%%
%%%%%		  Prepaired by Dr. R.M. Silva		%%%%%
%%%%%			Department of Statistics		%%%%%
%%%%%		  Faculty of Applied Sciences		%%%%%
%%%%%		University of Sri Jayewardenepura   %%%%%
%%%%%				Dec. 09th, 2017				%%%%%
%%%%%%%%%%%%%%%%%%%%%%%%%%%%%%%%%%%%%%%%%%%%%%%%%%%%%


% Please place the SJPLogo in the same folder that you are going to run this Latex file

\documentclass[a4paper,12pt]{article}
\usepackage{graphicx}
\usepackage{enumitem}
\usepackage{setspace}
\usepackage{multicol} 
\usepackage{color}
\usepackage[
top    = 2.00cm,
bottom = 2.00cm,
left   = 2.00cm,
right  = 2.00cm]{geometry}
\usepackage{hyperref}
\hypersetup{pdfstartview={XYZ null null 1.00}}

\begin{document}

\begin{figure}[ht]
	\begin{center}
		\includegraphics[angle=0,scale=0.05]{SJPLogo.jpg}
	\end{center}
\end{figure}

\vspace{-1cm}

\begin{center}
	University of Sri Jayewardenepura\\
	Faculty of Applied Scineces \\
	Department of Statistics
\end{center}


\noindent\rule{17cm}{0.4pt} 	%This will create a horizontal line

\begin{multicols}{3}
	\noindent\textbf{Batch: 2016/2017}
	
	\columnbreak
	\noindent\textbf{Year: 2020 }
	
	\columnbreak
	\noindent\textbf{Semester: First Semester}
	
\end{multicols}

\noindent\rule{17cm}{0.4pt}	% horizontal line

\vspace{0.5cm}
\noindent\textbf{Course Unit: STA 326 2.0 Programming and Data Analysis with R}\\

\noindent\textbf{Type of the course unit: Core for special degree students/ Optional for others}\\

\noindent\textbf{Pre-Requisites:}\\
\noindent\textit{{ STA 114 2.0 Probability and Distribution Theory I, STA 123 2.0 Probability and Distribution Theory II, STA 124 1.5 Data Analysis I, STA 226 1.5 Data Analysis II}}\\

\noindent\textbf{Workload:}\\
\noindent\textit{{ 100 learning hours. This includes approximately 30 hours of lectures and additional time spent by the student on self-learning, assignments and assessments.}}\\

\noindent\textbf{Course Objective(s):}
\begin{itemize}
	\setlength\itemsep{0.1mm}
	\item To introduce R programming for data science applications.
	\item 
\end{itemize}
\noindent\textbf{Course Contents:}
\begin{enumerate}[label*=\arabic*.]
	\setlength\itemsep{0.1mm}
	\item R basics: Objects in R, Data types, Operations, Installing packages, Control structures, Piping
	\item Writing functions in R
	\item Data analysis with the tidyverse
	\begin{enumerate}[label*=\arabic*.]
	\item Data import and export
	\item Data wrangling: Tidy data principles, Reshaping data into tidy form, Data transformation
	\item Data visualization: The grammar of graphics
	\item Statistical	modelling	and	inference
	\item Communication: Dynamic reproducible reporting
	\end{enumerate}
\end{enumerate}

\noindent\textbf{Learning Outcomes:}
At the end of this course, the student should be able to:
\begin{itemize}
	\setlength\itemsep{0.1mm}
	\item First learning outcome
	\item Second learning outcome
\end{itemize}

\noindent\textbf{Method of Assessment:}
\begin{itemize}
	\setlength\itemsep{0.1mm}
	\item Mid-semester examination: 20\%
	\item Final examination: 80\%
\end{itemize}

\noindent\textbf{Recommended Readings:}
\begin{itemize}
	\setlength\itemsep{0.1mm}
	\item Course website: Everything you want to know about the course, and everything you will need for the course (links to weekly reading, tutorials and lecture materials) will be posted at \textcolor{blue}{hellor.netlify.com}\\
	Author: Thiyanga Talagala
	\item Title: R for Data Science \\
		  Author(s): Hadley Wickham and Garrett Grolemund \\
		  Publisher: O'REILLY \\
		  This book is available online for free. Visit \textcolor{blue}{https://r4ds.had.co.nz/}
\end{itemize}

\noindent\textbf{Lecturer in Charge:}

\begin{itemize}
	\setlength\itemsep{0.1mm}
	\item[] Name: Dr Thiyanga  Talagala
	\item[] Email: ttalagala@sjp.ac.lk
\end{itemize}


\noindent\textbf{\textit{Note: The course can be dropped on or before xx.}}

\end{document}
